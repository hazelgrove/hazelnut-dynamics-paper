% !TEX root = hazelnut-dynamics.tex
\newcommand{\relatedWorkSection}{Related Work}
\section{\protect\relatedWorkSection} % don't like the all-caps thing that the template does, so protecting it from that
\label{sec:relatedWork}

\begin{itemize}
	\item Hazelnut paper
	\item simply typed underdeterminism paper
	\item work on partial evaluation and staging (incl. connections to modal logic -- ``modal analysis of staged computation'')
	\item symbolic evaluation
	\item full beta reduction
	\item Agda
	\item Idris
	\item GHC holes
	\item CMTT
	\item gradual typing (and dynamic typing)
		\begin{itemize}
			\item siek and taha paper
			\item snapl15 paper
			\item gradualizer paper
			\item maybe other things, e.g. several papers by ron garcia
		\end{itemize}
	\item DuctileJ stuff -- \url{https://homes.cs.washington.edu/~mernst/pubs/ductile-icse2011.pdf}
	\item OLEG from McBride's thesis
	\item Visual Studio (and others) support for edit-and-resume
	\item Scratch lets you just skip over statement holes
	\item prior work on confluence for the lambda calculus
	\item work on debuggers that allow you to inspect environments
		\begin{itemize}
		\item might be something relevant in the paper ``A Debugger for Standard ML'' 
		\item "Visualizing the evaluation of functional programs for debugging" by Whitington and Ridge
		\item "A lightweight interactive debugger for Haskell'' and ``Multiple-View Tracing for Haskell: a New Hat'' might be relevant
		\item ocamli -- \url{https://github.com/johnwhitington/ocamli}
		\item Better supporting debugging aids learning a novel programming language. -- Scaffidi at VLHCC 2017
		\item quote from Wadler in ``Why no one uses functional languages'':
			\begin{quote}
			“...there are few debuggers or
profilers for strict [functional] languages, perhaps because constructing them is not considered
research. This is a shame, since such tools are sorely needed, and there remains much of
interest to learn about their construction and use.
\end{quote}
		\end{itemize}
	\item papers that show up in a search for ``typed holes'' -- \url{https://scholar.google.com/scholar?hl=en&as_sdt=0%2C39&q=%22typed+holes%22&btnG=}
	\item maybe also search for ``partial programs'' 
	\item roly pererra work on program slicing
	\item should say something about how holes show up in program synthesis under various names (what?) 
	\item ``Achieving flexibility in direct-manipulation programming environments by relaxing the edit-time grammar'' -- \url{http://ieeexplore.ieee.org/document/1509511/}
	\item ``Call-by-value is dual to call-by-name'' might be relevant? \url{http://homepages.inf.ed.ac.uk/wadler/papers/dual/dual.pdf}
	\item mention how unspecified evaluation order is something that people do when talking about parallelism?
\end{itemize}
